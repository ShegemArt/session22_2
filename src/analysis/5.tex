\documentclass{article}

\usepackage[T2A, T1]{fontenc}
\usepackage[utf8]{inputenc}
\usepackage[russian]{babel}
\usepackage{defines}

\usepackage{amsmath, amsfonts, amssymb}
\usepackage{fullpage}

\begin{document}

\tickettitle{5}{Типы числовых множеств. Верхняя и нижняя грани множества.}

\define{типов числовых множеств}
\begin{enumerate}
	\item Отрезок (сегмент): $[a;b] := \{x\in\R\;|\;a\leq x\leq b\}$
	\item Интервал (открытый промежуток): $(a;b) := \{x\in\R\;|\;a<x<b\}$
	\item Окрестность ($\varepsilon$-окрестность) $a\in\R$: $(a-\varepsilon,a+\varepsilon)$
	\item Числовая прямая: $(-\infty;+\infty)$
	\item Полупрямая (луч): $[a;+\infty)$; $(-\infty;a]$
	\item Полуотрезок: $[a;b)$; $(a;b]$
	\item Открытая полупрямая (луч): $(a;+\infty)$; $(-\infty;a)$
	\item Расширенная числовая прямая: $\overline{\R}:=\R\cup\{-\infty,+\infty\}$
\end{enumerate}

\define{ограниченного множества}

$A\subset\R$ ограничено сверху, если $\exists M:\forall a\in A\;a<M$

$A\subset\R$ ограничено снизу, если $\exists M:\forall a\in A\;a>M$

$A\subset\R$ ограничено, если оно ограничено и сверху, и снизу

\define{точной верхней и нижней граней множества}

$M$ --- верхняя грань $A\subset\R$, если $\forall a\in A\;a\leq M$

$M$ --- нижняя грань $A\subset\R$, если $\forall a\in A\;a\geq M$

Наименьшая (наибольшая) из всех верхних (нижних) граней множества $A\subset\R$,
называется точной верхней (нижней) гранью.

Точная верхняя грань --- $\sup A$ (супремум)

Точная нижняя грань --- $\inf A$ (инфенум)

\theorem
\begin{align*}
	 & \exists M:\forall a\in A\;a<M \land A\neq\varnothing \Rarr \exists \sup A \\
	 & \exists M:\forall a\in A\;a>M \land A\neq\varnothing \Rarr \exists \inf A \\
\end{align*}

\end{document}
