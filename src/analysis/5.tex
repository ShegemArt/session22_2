\documentclass{article}

\usepackage[T2A, T1]{fontenc}
\usepackage[utf8]{inputenc}
\usepackage[russian]{babel}
\usepackage{defines}

\usepackage{fullpage}

\begin{document}

\tickettitle{5}{Типы числовых множеств. Верхняя и нижняя грани множества.}

\define{типов числовых множеств}
\begin{enumerate}
	\item Отрезок (сегмент): $[a;b] := \setdef{x\in\R}{a\leq x\leq b}$
	\item Интервал (открытый промежуток): $(a;b) := \setdef{x\in\R}{a<x<b}$
	\item Окрестность ($\eps$-окрестность) $a\in\R$:
	      $(a-\eps,a+\eps)$
	\item Числовая прямая: $(-\infty;+\infty)$
	\item Полупрямая (луч): $[a;+\infty)$; $(-\infty;a]$
	\item Полуотрезок: $[a;b)$; $(a;b]$
	\item Открытая полупрямая (луч): $(a;+\infty)$; $(-\infty;a)$
	\item Расширенная числовая прямая: $\overline{\R}:=\R\cup\{-\infty,+\infty\}$
\end{enumerate}

\define{ограниченного множества}

$A\subset\R$ ограничено сверху, если $\exists M:\forall a\in A\;a<M$

$A\subset\R$ ограничено снизу, если $\exists M:\forall a\in A\;a>M$

$A\subset\R$ ограничено, если оно ограничено и сверху, и снизу

\define{точной верхней и нижней граней множества}

$M$ --- верхняя грань $A\subset\R$, если $\forall a\in A\;a\leq M$

$M$ --- нижняя грань $A\subset\R$, если $\forall a\in A\;a\geq M$

Наименьшая (наибольшая) из всех верхних (нижних) граней множества $A\subset\R$,
называется точной верхней (нижней) гранью.

Точная верхняя грань --- $\sup A$ (супремум)

Точная нижняя грань --- $\inf A$ (инфенум)

\pagebreak

\theorem

$A$ --- ограничено сверху $\Rarr \exists \sup A$

$A$ --- ограничено снизу $\Rarr \exists \inf A$

\proof
\begin{enumerate}
	\item$A$ --- ограничено сверху

	Рассмотрим $B:=\setdef{x\in\R}{x>a\;\forall a\in A}$.

	Заметим, что $B$ --- не множество всех верхних граней $A$.
	\begin{align}
		 & B:=\setdef{x\in\R}{x>a\;\forall a\in A}\notag                                       \\
		 & B':=\R\setminus B\Rarr (B'\cup B=\R) \land (B'\cap B=\eset)\label{sepB}             \\
		 & (\forall a \in A)\;(a\notin B)\land(a\in\R) \Rarr a\in B'\notag                     \\
		 & (\forall b\in B)(\forall x\in\R)\;x>b\Rarr(\forall a\in A)\;x>b>a\Rarr x\in B\notag \\
		 & x\in B'\Rarr x\in\R\land x\notin B\Rarr (\forall b\in B)\;x\leq b\notag             \\
		 & (\forall b\in B)(\forall b'\in B')\;b'\leq b\label{diffB}
	\end{align}
	По 17й аксиоме $\R$, сформулированной через Дедекиндовы сечения:
	\begin{align}
		\eqref{sepB}\land\eqref{diffB}\Rarr \exists c\in\R:
		(\forall b\in B)(\forall b'\in B')\;b'\leq c\leq b\Rarr
		(\forall b\in B)(\forall a\in A)\;a\leq c\leq b\label{defc}
	\end{align}

	Рассмотрим $U:=\setdef{x\in\R}{x\geq a\;\forall a\in A}$.

	$U$ --- множество всех верхних граней $A$. Если $\exists\min U$, то $\sup A:=\min U$.

	Рассмотрим теперь $U\setminus B$, пусть $U\setminus B\neq\eset$:
	\begin{align}
		m\in(U\setminus B)
		 & \Rarr m\in U\land m\notin B \Rarr (m\geq a\;\forall a \in A)\land(\exists a\in A:m\leq a)\Rarr\notag \\
		 & \Rarr(m\geq a\;\forall a \in A)\land(m\in A)\notag                                                   \\
		x\in B
		 & \Lrarr x>a\;\forall a\in A\Lrarr x>m\geq a\;\forall a\in A\Lrarr x>m\label{Bdef}
	\end{align}
	Из $\eqref{Bdef}$ следует, что $B=(m;+\infty)\land B'=(-\infty;m] \Rarr m=c$

	Таким образом, если $U\setminus B\neq \eset$, то $U\setminus B=\{c\}\Rarr U\setminus B\subset \{c\}$
	\begin{align*}
		 & (U\setminus B \subset \{c\}) \land (c\in U \text{ следует из $\eqref{defc}$}) \Rarr U=B\cup \{c\} \\
		 & (c\leq u\;\forall u\in U)\land (c\in U) \Rarr \min U=c\Rarr \sup A=c\;\square.
	\end{align*}

	\item$A$ --- ограничено снизу: аналогично
\end{enumerate}

\end{document}
