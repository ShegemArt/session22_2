\documentclass{article}

\usepackage[T2A, T1]{fontenc}
\usepackage[utf8]{inputenc}
\usepackage[russian]{babel}
\usepackage{defines}

\usepackage{fullpage}

\begin{document}

\tickettitle{50}{Теорема Ролля}

\theorem

$f$ непрерывна на $[a;b]$ и дифференцируема на $(a;b)$
\begin{align*}
	f(a)=f(b) \Rarr \exists \gamma\in(a;b):f'(\gamma)=0
\end{align*}

\proof

По теореме Вейерштрасса $f$ ограничена на $[a;b]$ и она достигает своих максимума и минимума:
\begin{align*}
	M & :=\sup f([a;b]) & \exists x_M\in[a;b] & :f(x_M)=M \\
	m & :=\inf f([a;b]) & \exists x_m\in[a;b] & :f(x_m)=m
\end{align*}
\begin{enumerate}
	\item$M=m$

	Тогда $f(x)=const \Rarr (\forall\gamma\in(a;b))\;f'(\gamma)=0\qed$

	\item$M\neq m$

	И минимум, и максимум достигаются функцией на $[a;b]$.

	$f(a)=f(b) \Rarr$ хотя бы один из них достигается на $(a;b)$, там будет точка экстремума $f$:
	\begin{align*}
		\exists\gamma\in(a;b):\gamma\text{ --- точка экстремума $f$} \Rarr f'(\gamma)=0\qed
	\end{align*}
\end{enumerate}

\end{document}
