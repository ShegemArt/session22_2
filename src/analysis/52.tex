\documentclass{article}

\usepackage[T2A, T1]{fontenc}
\usepackage[utf8]{inputenc}
\usepackage[russian]{babel}
\usepackage{defines}

\usepackage{fullpage}

\raggedright
\begin{document}

\tickettitle{52}{Теорема Коши}

\theorem

\begin{enumerate}
	\item$f$ непрерывна на $[a;b]$ и дифференцируема на $(a;b)$
	\item$g$ непрерывна на $[a;b]$ и дифференцируема на $(a;b)$
	\item$g'(x)\neq0\;\forall x\in (a;b)$
\end{enumerate}
\begin{align*}
	\exists \gamma\in(a;b):\frac{f'(\gamma)}{g'(\gamma)}=\frac{f(b)-f(a)}{g(b)-g(a)}
\end{align*}

\proof
\begin{align*}
	 & F(x):=f(x)-\lambda g(x) \land F(a)=F(b)             \\
	 & F(a)=F(b) \Rarr f(a)-\lambda g(a)=f(b)-\lambda g(b)
	\Rarr\lambda=\frac{f(b)-f(a)}{g(b)-g(a)}
\end{align*}

$F$ непрерывна на $[a;b]$ и дифференцируема на $(a;b)$, тк $f$ и $g$ непрерывны на $[a;b]$ и дифференцируемы на $(a;b)$. Тогда, по теореме Ролля:
\begin{align*}
	\exists\gamma\in(a;b):F'(\gamma)=0 \Rarr f'(\gamma)-\lambda g'(\gamma)=0
	\Rarr \frac{f'(\gamma)}{g'(\gamma)}=\frac{f(b)-f(a)}{g(b)-g(a)}\qed
\end{align*}

\end{document}
