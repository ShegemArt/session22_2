\documentclass{article}

\usepackage{defines}

\begin{document}

\tickettitle{35}{Теорема Вейерштрасса}

\theorem[Вейерштрасса]

$f$ --- непрерывна на $[a;b]$ $\Rarr$ $f$ --- ограничена на $[a;b]$ и достигает
своих минимума и максимума

\proof

\begin{enumerate}
	\item$f$ --- ограничена на $[a;b]$, если она непрерывна на $[a;b]$

	$f$ равномерно непрерывна на $[a;b]$ (по т. Кантора):
	\begin{align*}
		 & \llet\eps=1\;\exists\delta>0:(\forall x,x'\in [a;b]:(|x-x'|<\delta))\;|f(x)-f(x')|<1
	\end{align*}

	Разделим $[a;b]$ на $n$ частей: $\frac{b-a}{n}<\delta$
	\begin{align*}
		a_0 & :=a & a_{i+1}-a_i & =\frac{b-a}{n} \\
		a_n & :=b & a_{i+1}-a_i & <\delta
	\end{align*}
	\begin{align*}
		 & |a_{i+1}-a_i|<\delta  \Rarr |f(x)-f(a_i)|<1\;\forall x\in[a_{i-1};a_i]
		\Rarr |f(x)|<1+|f(a_i)|\;\forall x\in[a_{i-1};a_i]                                 \\
		 & A:=\max\setdef{1+|f(a_i)|}{i=\overline{1,n}} \Rarr |f(x)|<A\;\forall x\in [a;b]
		\Rarr f\text{ --- ограничена на $[a;b]$}
	\end{align*}

	\item$f$ достигает своих минимума и максимума

	$f$ --- ограничена на $[a;b]$ $\Rarr$ $\exists M:=\sup f([a;b])$, $\exists m:=\inf f([a;b])$

	Покажем, что $M$ и $m$ достигаются на $[a;b]$. Пойдем от противного, пусть это не так, тогда:
	\begin{align*}
		 & \llet (\forall x\in[a;b])\;f(x)<M\Rarr\llet g(x):=\frac{1}{M-f(x)}\text{ --- непрерывна на $[a;b]$}\Rarr \\
		 & \Rarr g(x) \text{ --- ограничена на $[a;b]$ } \Rarr (\forall x\in [a;b])\;\exists L:g(x)<L\Rarr          \\
		 & \Rarr \frac{1}{M-f(x)}<L\Rarr f(x)<M-\frac{1}{L}\text{, но $M=\sup f([a;b])$}\qed
	\end{align*}
	\begin{align*}
		 & \llet (\forall x\in[a;b])\;f(x)>m\Rarr\llet g(x):=\frac{1}{f(x)-m}\text{ --- непрерывна на $[a;b]$}\Rarr \\
		 & \Rarr g(x) \text{ --- ограничена на $[a;b]$ } \Rarr (\forall x\in [a;b])\;\exists L:g(x)<L\Rarr          \\
		 & \Rarr \frac{1}{f(x)-m}<L\Rarr f(x)>m+\frac{1}{L}\text{, но $m=\inf f([a;b])$}\qed
	\end{align*}
\end{enumerate}

\end{document}
