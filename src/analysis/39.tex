\documentclass{article}

\usepackage{defines}

\usepackage{enumitem}

\begin{document}

\tickettitle{39}{Условие непрерывности монотонной функции}

\theorem

$f$ --- монотонна на промежутке $X$
\begin{align*}
	f(X)=Y\text{, где $Y$ --- промежуток}\Rarr f\text{ --- непрерывна на $X$ }
\end{align*}

Здесь под промежутком имеется в виду как открытый, полуоткрытый, так и закрытый промежуток

\proof

\newcommand\Xcut{\widetilde{X}}
\newcommand\Xleft{\Xcut_L}
\newcommand\Xright{\Xcut_R}

Без ограничения общности пусть $f$ --- монотонно возрастающая.
Докажем, что $f$ --- непрерывна слева в
$\Xcut:=\setdef{x\in X}{\exists x'\in X:x'<x}=X\setminus\setdef{x\in X}{\forall x'\in X:x\leq x'}$.

$\Xcut=X\setminus\{\min X\}$, если $\exists\min X$. $\Xcut=X$, если его не существует.

Иными словами:

\begin{table}[h]
	\centering
	\begin{tabular}{c|c}
		$X$     & $\Xcut$ \\
		\hline            \\[-0.8em]
		$(a;b)$ & $(a;b)$ \\[0.3em]
		$(a;b]$ & $(a;b]$ \\[0.3em]
		$[a;b)$ & $(a;b)$ \\[0.3em]
		$[a;b]$ & $(a;b]$
	\end{tabular}
\end{table}

Докажем, что $f$ --- непрерывна слева в $\forall x_0\in\Xcut$:

Для этого будем рассматривать $f$ на $\Xleft:=\setdef{x\in\hat{X}}{x<x_0}$ и
$\Xright:=\setdef{x\in\hat{X}}{x_0\geq x}$

\begin{enumerate}
	\item{}Найдём $\slim{x\to x_0-0}f(x)$:
	\begin{align*}
		 & (\forall x\in\Xleft)\;f(x)\leq f(x_0)\Rarr \exists\sup f(\Xleft)          \\
		 & g:=\sup f(\Xleft)
		\Rarr \forall\eps>0\;\exists\delta>0:f(x_0-\delta)>g-\eps\Rarr               \\
		 & \Rarr (\forall x\in(x_0-\delta;x_0))\;g-\eps<f(x_0-\delta)\leq f(x)\leq g
		\Rarr g-\eps<f(x)\leq g\Rarr \slim{x\to x_0-0}f(x)=g
	\end{align*}

	\item{}Докажем, что $f(x_0)=g$
	\begin{align}
		\llet f(x_0)<g & \Rarr \llet\eps=g-f(x_0)>0\;\exists x\in\Xleft:g-\eps<f(x)\Rarr f(x_0)<f(x)\text{, но } x<x_0\label{39:flg} \\
		\llet f(x_0)>g & ,\; (f(x)\leq g\;\forall x\in\Xleft)\land(f(x_0)\leq f(x)\;\forall x\in\Xright)\Rarr\notag                  \\
		               & \Rarr (g;f(x_0))\not\subset Y\text{, но $Y$ --- промежуток}\label{39:fgg}
	\end{align}
	\begin{align*}
		\eqref{39:flg}\land\eqref{39:fgg}\Rarr f(x_0)=g
	\end{align*}
\end{enumerate}

Таким образом, была доказана непрерывность слева для внутренних точек $X$ и $\max X$, если он существует.
Аналогично доказывается непрерывность справа для внутренних точек $X$ и $\min X$, если он существует.

Таким образом для любой внутренней точки $x_0$ промежутка $X$:
\begin{align*}
	\slim{x\to x_0-0}f(x)=f(x_0)\land\slim{x\to x_0+0}f(x)=f(x_0)\Rarr\slim{x\to x_0}f(x)=f(x_0)
\end{align*}

Из этого следует, что $f$ --- непрерывна на $X\qed$

\end{document}
