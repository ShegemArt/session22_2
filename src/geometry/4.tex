\documentclass{article}

\usepackage{defines}

\begin{document}

\tickettitle{4}{Базис линеала. Теорема о единственном представлении вектора по базису. Теорема~о~формировании~координат~суммы~векторов~и~произведения~вектора~на~число. Теорема~о~линейной~зависимости~n~+~1~элемента~линеала.}

\define{базиса}

$(e_1,e_2,...,e_n)\subset L$ --- базис линеала $L$, если:
\begin{enumerate}
	\item{}$(e_1,e_2,...,e_{n})$ --- лнз
	\item{}$(e_1,e_2,...,e_{n})$ --- упорядоченный набор
	\item{}$(\forall x\in L)\;\exists\alpha_{i}\in\R:x=\displaystyle\sum_{i=1}^{n}\alpha_{i}e_{i}$
\end{enumerate}

Коэфиценты разложения по базису называются координатами элемента в базисе.

\theorem

Разложение $x\in L$ по базису $(e_1,e_2,...,e_n)$ единственно.

\proof

Допустим, что существует 2 разложения $\gamma_{i}$ и $\mu_{i}$:
\begin{align*}
	 & x=\sum_{i=1}^{n}\gamma_{i}e_{i} &  & x=\sum_{i=1}^{n}\mu_{i}e_{i}
\end{align*}
\begin{align*}
	 & x-x=\sum_{i=1}^{n}\gamma_{i}e_{i}-\sum_{i=1}^{n}\mu_{i}e_{i}\Rarr\sum_{i=1}^{n}(\gamma_{i}-\mu_{i})e_{i}=0\Rarr \\
	 & \Rarr(\forall i=\overline{1,n})\;\gamma_{i}-\mu_{i}=0\Rarr\gamma_{i}=\mu_{i}\text{ (по лнз базиса)}
\end{align*}

Таким образом, любые 2 разложения равны, значит существует только одно разложение$\qed$

\theorem

$(e_1,e_2,...,e_{n})$ --- базис $L$

$\alpha_{i}$ --- координаты $\vec{a}$

$\beta_{i}$ --- координаты $\vec{b}$
\begin{align*}
	 & \vec{c}=\vec{a}+\vec{b}\Rarr(\alpha_{i}+\beta_{i})\text{ --- координаты $\vec{c}$}                     &
	 & (\forall \lambda\in\R)\;\vec{c}=\lambda\vec{a}\Rarr(\lambda\alpha_{i})\text{ --- координаты $\vec{c}$}
\end{align*}

\proof
\begin{align*}
	 & \vec{c}=\vec{a}+\vec{b}\Rarr\vec{c}=\sum_{i=1}^{n}\alpha_{i}e_{i}+\sum_{i=1}^{n}\beta_{i}e_{i}=
	\sum_{i=1}^{n}(\alpha_{i}+\beta_{i})e_{i}\Rarr(\alpha_{i}+\beta_{i})\text{ --- разложение $\vec{c}$} \\
	 & \vec{c}=\lambda\vec{a}\Rarr\vec{c}=\lambda\sum_{i=1}^{n}\alpha_{i}e_{i}=
	\sum_{i=1}^{n}(\lambda\alpha_{i})e_{i}\Rarr(\lambda\alpha_{i})\text{ --- разложение $\vec{c}$}
\end{align*}

\pagebreak

\theorem

Каждый из $\{y_1,y_2,...,y_{n+1}\}$ представим через лнз систему $\{e_1,e_2,...,e_{n}\}$ $\Rarr$
$\{y_1,y_2,...,y_{n+1}\}$ --- лз.

\proof

Доказательство по индукции ($P(n)$ --- верность теоремы для $n$):
\begin{enumerate}
	\item{}$P(1)$

	Представим $y_{i}$ через $e_1$:
	\begin{align*}
		 & y_1=\lambda_1e_1 &  & y_2=\lambda_2e_1
	\end{align*}

	\begin{enumerate}
		\item{}$\exists\lambda_{i}=0\Rarr y_{i}=0\Rarr \{y_1,y_2\}\text{ содержит $0$}\Rarr\{y_1,y_2\}\text{ --- лз}\qed$
		\item{}$\forall\lambda_{i}\neq 0$
		\begin{align*}
			 & \lambda_2y_1+(-\lambda_1)y_2=(\lambda_2\lambda_1)e_1+(-\lambda_1\lambda_2)e_1=0 &  & (\lambda_2)^{2}+(-\lambda_1)^{2}\neq 0\qed
		\end{align*}
	\end{enumerate}

	\item{}$P(n-1)\Rarr P(n)$

	Выразим $y_{j}$:
	\begin{align*}
		y_{j}=\sum_{i=1}^{n}\lambda_{ij}e_{i} \quad j=\overline{1,n+1}
	\end{align*}

	\begin{enumerate}
		\item{}$\exists y_{i}=0\Rarr\{y_1,y_2,...,y_{n}\}$ --- лз
		\item{}$\forall y_{i}\neq 0$

		Без ограничения общности пусть $\lambda_{11}\neq 0$ (можно предположить, тк $y_{1}\neq 0$).

		Введём $z_{k}$ для $k=\overline{2,n+1}$:
		\begin{align*}
			z_{k} & :=y_{k}-\frac{\lambda_{1k}}{\lambda_{11}}y_{1}=\sum_{i=1}^{n}\lambda_{ik}e_{i}-\sum_{i=1}^{n}\frac{\lambda_{1k}}{\lambda_{11}}\lambda_{1i}e_{i}
			=(\lambda_{1k}-\frac{\lambda_{1k}}{\lambda_{11}}\lambda_{11})e_{1}+\sum_{i=2}^{n}\left(\lambda_{ik}-\frac{\lambda_{1k}}{\lambda_{11}}\lambda_{1i}\right)e_{i}= \\
			      & =0\cdot e_{i}+\sum_{i=2}^{n}\left(\lambda_{ik}-\frac{\lambda_{1k}}{\lambda_{11}}\lambda_{1i}\right)e_{i}
			=\sum_{i=2}^{n}\left(\lambda_{ik}-\frac{\lambda_{1k}}{\lambda_{11}}\lambda_{1i}\right)e_{i}
		\end{align*}

		$n$ векторов $z_{k}$ были выражены через $n-1$ векторов $e_{i}$ $(i=\overline{2,n})\Rarr z_{k}$ --- лз (по $P(n-1)$):
		\begin{align*}
			 & \exists\mu_{i}:\sum_{k=2}^{n}\mu_{k}z_{k}=\left(-\sum_{k=2}^{n}\mu_{k}\frac{\lambda_{1k}}{\lambda_{11}}\right)y_1+\sum_{k=2}^{n}\mu_{k}y_{k}=0 \\
			 & \left(-\sum_{k=2}^{n}\mu_{k}\frac{\lambda_{1k}}{\lambda_{11}}\right)^{2}+\sum_{k=2}^{n}\mu_{k}^{2}\neq 0\qed
		\end{align*}
	\end{enumerate}

\end{enumerate}

\end{document}
